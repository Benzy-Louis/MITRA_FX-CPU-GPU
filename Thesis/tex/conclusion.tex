\chapter{Conclusions, and future developments}
\label{ConclusionOnDissertation}
\lhead{\emph{Conclusions, and future developments}} 
To conclude this report we might say that the field of Imaging in Radio Astronomy is a non-exhaustive one, it is interesting to appreciate that there are an innumerable set of factors which come into the play and sum up as a whole to produce an end-result. We have approached the subject from the viewpoint of the Signal \& Image Processing module where we have focused on physical and extensively mathematical descriptions which would enable one to have a clue on how the data is to be processed, however even in this scope we have certainly not taken the most effective route, there is still much improvement to be done in this report to make something more comprehensive and self-supporting on the subject to avoid unexplained areas, or to reduce the focus to more effective areas in the scope of the module. An improvement could be the inclusion of actual practical examples for the reader to practice with simple generated or raw data by using the acquired techniques and knowledge during the course, so that in this way the unexplained areas mostly concerning the use of particular mathematical functions can be easily grasped. To wrap up we thus encourage the target readers to consult the literature, where everything is extensively explained and the scope of this field does not end at image processing, the broader field of physics behind is something that we encourage the reader to look for. 