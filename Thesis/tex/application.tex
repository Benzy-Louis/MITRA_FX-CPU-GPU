\section{Application to other Strange Attractors}
\subsection{The Lorenz attractor }
The Lorenz system is a system of ordinary differential equations (the Lorenz equations) first studied by Edward Lorenz. It is notable for having chaotic solutions for certain parameter values and initial conditions. In particular, the Lorenz attractor is a set of chaotic solutions of the Lorenz system.
In 1963, Edward Lorenz developed a simplified mathematical model for atmospheric convection. The model is a system of three ordinary differential equations now known as the Lorenz equations:
\begin{subequations}
\begin{align}
\frac{\mathrm{d}x}{\mathrm{d}t} &= \sigma(y-x),  \\
\frac{\mathrm{d}y}{\mathrm{d}t} &= x(\rho -z) - y,\\
\frac{\mathrm{d}z}{\mathrm{d}t} &= xy-\beta z
\end{align}
\end{subequations}
Here $x$, $y$, and $z$ make up the system state, $t$ is time, and $\sigma$, $\rho$, $\beta$ are the system parameters.

The system exhibits chaotic behavior at $ \sigma = 10$, $\beta = 8/3$, and $\rho = 28$ . To compute the fractal dimension, of this attractor these parameters have being used and the orbit generated by solving the system of equations using the Runge Kutta 4th order numerical approximation with the initial values of $x(0) = 1,$ $y(0) = 1,$ and $z(0)=1$ and uniform steps of $0.01$.  


\begin{center}
  \includegraphics[width = 0.60\textwidth]{img/lorenz}
  \captionof{figure}{The Lorenz attractor $ \sigma = 10$, $\beta = 8/3$, and $\rho = 28$}
	\label{lorenzatt:fig}
\end{center}

\newpage
\lstinputlisting{Function_files/lorenzRK.m}
\lstinputlisting{Function_files/LorenzSolver.m}
\lstinputlisting{Function_files/lorenzEq.m}
The strange attractor being embedded in a 3{-}D space, the algorithms have been implemented to compute using the 3 components. A large range of $\varepsilon$ is first selected to locate the scaling region, for the Lorenz attractor a logarithmically spaced interval from \textbf{e$^{-10}$} to \textbf{e$^{6}$} is used by typing in the command window. 
\begin{verbatim}
>> epsi = exp(linspace(-10,6,100))';
\end{verbatim}
The points are generated in the array [\textbf{r}] then the values of $N(\varepsilon)$ are computed using the function as follows:
\begin{verbatim}
>> N_E = L3capacitydimension(r(1:3e4,:),epsi);
\end{verbatim}
Here with this command $N(\varepsilon)$ is computed for the first~$3\times10^{4}$ points, the plots of $\log{N(\varepsilon)}$ vs $\log{\frac{1}{\varepsilon}}$ are then done as in Figure \ref{lorenzcap:fig} with different number of points to inspect the scaling region. The values of $\varepsilon$ from \textbf{e$^{-0.5}$} to \textbf{e$^{1}$} is selected as the scaling region for $n = 3\times10^{4}$ points, to investigate how the value of the slope, i.e. the estimate of the capacity dimension, $D_{c}$, changes as $n$ is increased, the calculation of $N(\varepsilon)$ is done with several number of points, $n$, within that interval with 100 logarithmically spaced values of $\varepsilon$. The least square method is then used to calculate the slope of the $\log{N(\varepsilon)}$ vs $\log{\frac{1}{\varepsilon}}$ plot, the values are tabulated in Table~\ref{lorcap:tb}. As the number of points used from $n = 3 \times 10^{4}$ to $n = 1 \times 10^{7}$ the slope becomes steeper but does not seem to settle to a definite value, the estimate of $D_{c} \approx 1.953$ at $n = 1 \times 10^{7}$ is not in agreement with the of the value of 2.06 $\pm$ 0.01 (Grassberger and Procaccia 1983).
\begin{center}
\includegraphics[width = 4in]{img/lorenzcap}
\captionof{figure}{ $\log{N(\varepsilon)}$ vs $\log{\frac{1}{\varepsilon}}$ with the Lorenz attractor}
\label{lorenzcap:fig}
\end{center}

\begin{center}
\begin{tabular}{ |l|l| }
\hline
\multicolumn{1}{|c|}{\multirow{3}{*}{$n$}} & \multicolumn{1}{|c|}{$D_{c} $}\\
\cline{2-2}
&\multicolumn{1}{|c|}{100 intervals, logarithmically spaced range of $\varepsilon$ used}\\
\cline{2-2}
 &\rule{0pt}{12pt} \hspace{85pt} $[\boldsymbol{e}^{-0.5},\boldsymbol{e}^{1}]$ \\
\hline
$3\times 10^{4}$ & \hspace{80pt} 1.748 $\pm$ 0.004  \\
\hline
$6\times 10^{4}$ & \hspace{80pt} 1.802 $\pm$ 0.003  \\
\hline
$1\times 10^{5}$ & \hspace{80pt} 1.850 $\pm$ 0.002   \\
\hline
$1\times 10^{6}$ & \hspace{80pt} 1.932 $\pm$ 0.001   \\
\hline
$1\times 10^{7}$ & \hspace{80pt} 1.953 $\pm$ 0.001 \\
\hline
\end{tabular}
\captionof{table}{Calculation of capacity dimension of Lorenz attractor, least-square fittings\label{lorcap:tb}}
\end{center}

When the information entropy, $S(\varepsilon)$ is calulated over the same range using the written function,
\begin{verbatim}
>> S_E = L3informationdimension(r,epsi);
\end{verbatim}
 The plot $S(\varepsilon)$ vs $\log{\frac{1}{\varepsilon}}$ yields Figure~\ref{lorenzinf:fig}. Over the same scaling region as previously (\textbf{e$^{-0.5}$} to \textbf{e$^{1}$}) the slope of the plot of $S(\varepsilon)$ vs $\log{\frac{1}{\varepsilon}}$ estimates the information, $D_{i} \approx 2.04$ when $n = 1 \times 10^{7}$ points are used.  
\begin{verbatim}
>> C_E = L3correlationdimension(r,epsi);
\end{verbatim}
The computation of the correlation integral of the Lorenz Attractor over the whole range, shows a linear scaling on an appreciable range, calculating the slope of the plot in Figure~\ref{lorenzcor:fig} over the region from \textbf{e$^{-1.8}$} to \textbf{e$^{1.3}$} (Table ~\ref{lorcor:tb})  gives an estimate of the correlation dimension, $D_{cor} \approx 2.06$ which is close to the value of $2.05 \pm 0.01$ from Grassberger and Procaccia 1983.
\begin{figure}[h]
\begin{center}
\includegraphics[width = 3.9in]{img/lorenzinf}
\caption{$S(\varepsilon)$ vs $\log{\frac{1}{\varepsilon}}$ with the Lorenz attractor}
\label{lorenzinf:fig}
\end{center}
\end{figure}
\begin{center}
\includegraphics[width = 3.8in]{img/lorenzcor}
\captionof{figure}{ $\log{C(\varepsilon)}$ vs $\log{{\varepsilon}}$ with the Lorenz attractor}
\label{lorenzcor:fig}
\end{center}


\begin{center}
\begin{tabular}{ |l|l| }

\hline
\multicolumn{1}{|c|}{\multirow{3}{*}{$n$}} & \multicolumn{1}{|c|}{$D_{i} $}\\
\cline{2-2}
&\multicolumn{1}{|c|}{100 intervals, logarithmically spaced range of $\varepsilon$ used}\\
\cline{2-2}
 &\rule{0pt}{12pt} \hspace{85pt} $[\boldsymbol{e}^{-0.5},\boldsymbol{e}^{1}]$ \\
\hline
$3\times 10^{4}$ & \hspace{80pt} 1.958 \hspace{5pt}$\pm$ 0.003  \\
\hline
$6\times 10^{4}$ & \hspace{80pt} 1.991 \hspace{5pt}$\pm$ 0.001  \\
\hline
$1\times 10^{5}$ & \hspace{80pt} 2.014 \hspace{5pt}$\pm$ 0.001   \\
\hline
$1\times 10^{6}$ & \hspace{80pt} 2.0404 $\pm$ 0.0005   \\
\hline
$1\times 10^{7}$ & \hspace{80pt} 2.0437 $\pm$ 0.0005  \\
\hline
\end{tabular}
\end{center}
\captionof{table}{Calculation of information dimension of Lorenz attractor, least-square fittings\label{lorinf:tb}}



\begin{center}
\begin{tabular}{ |l|l| }
\hline
\multicolumn{1}{|c|}{\multirow{3}{*}{$n$}} & \multicolumn{1}{|c|}{$D_{cor} $}\\
\cline{2-2}
&\multicolumn{1}{|c|}{100 intervals, logarithmically spaced range of $\varepsilon$ used}\\
\cline{2-2}
 &\rule{0pt}{12pt}  \hspace{85pt} $[\boldsymbol{e}^{-1.8},\boldsymbol{e}^{1.3}]$  \\
\hline
$3\times 10^{4}$ & \hspace{80pt} 2.098  \hspace{4pt} $\pm$ 0.003  \\
\hline
$5\times 10^{4}$ & \hspace{80pt} 2.0640 $\pm$ 0.0006 \\
\hline
$6\times 10^{4}$ & \hspace{80pt} 2.0621 $\pm$ 0.0006 \\
\hline
\end{tabular}
\captionof{table}{Calculation of the correlation dimension of Lorenz attractor, least-square fittings\label{lorcor:tb}}
\end{center}

\subsection{The Double scroll-attractor}
The double-scroll attractor also known as Chua's attractor is a strange attractor observed from a physical electronic chaotic circuit with a single nonlinear resistor . The double-scroll system is often described by a system of three nonlinear ordinary differential equations and a 3-segment piecewise-linear equation:

\begin{subequations}
\begin{align}
\frac{\mathrm{d}x}{\mathrm{d}t} &= \alpha(y- \phi(x)),  \\
\frac{\mathrm{d}y}{\mathrm{d}t} &= x - y + z, \\
\frac{\mathrm{d}z}{\mathrm{d}t} &= \beta y.\\
\phi(x) &= bx + \frac{1}{2}(a - b)\left[|x + 1| - |x - 1|\right]
\end{align}
\end{subequations}
The system exhibits chaotic behavior at $ \alpha = 15.6$, $\beta = 28$, $a = -8/7$, and $b = -5/7$.\\
To compute the fractal dimension, of this attractor the parameters have being used with the initial values of $x(0) = 0.7,$ $y(0) = 0,$ and $z(0)=0$, and the points generated by solving the system of equations using the Runge Kutta 4th order numerical approximation with even steps of 0.01.
	\begin{center}
	\includegraphics[width = 0.7\textwidth]{img/chua}
	\captionof{figure}{The Double-Scroll attractor}
	\label{Chuaatt:fig}
	\end{center}
	\vspace{10pt}
\lstinputlisting[caption= Script file ChuaRk.m]{Function_files/ChuaRK.m}
\vspace{70pt}
\lstinputlisting[caption= Function file ChuaSolver.m]{Function_files/ChuaSolver.m}
\lstinputlisting[caption= Function file Chua.m]{Function_files/Chua.m}

	\begin{center}
	\includegraphics[width = 5in]{img/chuacap}
	\captionof{figure}{ $\log{N(\varepsilon)}$ vs $\log{\frac{1}{\varepsilon}}$ with the Double{-}Scroll attractor}
	\label{chuacap:fig}
	\end{center}
As the Double-Scroll attractor is concerned a suitable region in the plot of Figure~\ref{chuacap:fig} is chosen as the range from \textbf{e$^{-3}$} to \textbf{e$^{-1}$}, the estimate of the capacity dimension, $D_{c}$, increases substantially as the number of points used in the computation is increased as seen in the table ~\ref{chuacap:tb}.

\begin{center}
\begin{tabular}{ |l|l| }
\hline
\multicolumn{1}{|c|}{\multirow{3}{*}{$n$}} & \multicolumn{1}{|c|}{$D_{c} $}\\
\cline{2-2}
&\multicolumn{1}{|c|}{100 intervals, logarithmically spaced range of $\varepsilon$ used}\\
\cline{2-2}
 &\rule{0pt}{12pt}  \hspace{85pt} $[\boldsymbol{e}^{-3},\boldsymbol{e}^{-1}]$  \\
\hline
$3\times 10^{4}$ & \hspace{80pt} 1.904 $\pm$ 0.004  \\
\hline
$6\times 10^{4}$ & \hspace{80pt} 1.948 $\pm$ 0.003 \\
\hline
$1\times 10^{5}$ & \hspace{80pt} 1.974 $\pm$ 0.003 \\
\hline
$1\times 10^{6}$ & \hspace{80pt} 2.128 $\pm$ 0.004  \\
\hline
$1\times 10^{7}$ & \hspace{80pt} 2.188 $\pm$ 0.002  \\
\hline
\end{tabular}
\captionof{table}{Calculation of Capacity dimension of Chua Double-Scroll attractor, least-square fittings\label{chuacap:tb}}
\end{center}


	\begin{center}
    \includegraphics[width = 5in]{img/chuainf}
	\captionof{figure}{ ${S(\varepsilon)}$ vs $\log{\frac{1}{\varepsilon}}$ with the Double{-}Scroll attractor}
	\label{chuainf:fig}
	\end{center}
Computation of the information dimension,$D_{i}$ over the same range yields more moderate estimates as seen in the table ~\ref{chuainf:tb}, with an estimates of $D_{i} \approx 1.998$ with $n = 1 \times 10^{7}$.
\begin{center}
\begin{tabular}{ |l|l| }

\hline
\multicolumn{1}{|c|}{\multirow{3}{*}{$n$}} & \multicolumn{1}{|c|}{$D_{i} $}\\
\cline{2-2}
&\multicolumn{1}{|c|}{100 intervals, logarithmically spaced range of $\varepsilon$ used}\\
\cline{2-2}
 &\rule{0pt}{12pt}  \hspace{85pt} $[\boldsymbol{e}^{-3},\boldsymbol{e}^{-1}]$  \\
\hline
$3\times 10^{4}$ & \hspace{80pt} 1.830 $\pm$ 0.008  \\
\hline
$6\times 10^{4}$ & \hspace{80pt} 1.913 $\pm$ 0.006 \\
\hline
$1\times 10^{5}$ & \hspace{80pt} 1.977 $\pm$ 0.006 \\
\hline
$1\times 10^{6}$ & \hspace{80pt} 1.996 $\pm$ 0.003  \\
\hline
$1\times 10^{7}$ & \hspace{80pt} 1.998 $\pm$ 0.002  \\
\hline
\end{tabular}
\captionof{table}{Calculation of Information dimension of Chua Double-Scroll attractor, least-square fittings\label{chuainf:tb}}
\end{center}

	\begin{center}
	\includegraphics[width = 4in]{img/chuacor}
	\captionof{figure}{ $\log{C(\varepsilon)}$ vs $\log{{\varepsilon}}$ with the Double{-}Scroll attractor}
	\label{chuacor:fig}
	\end{center}


\begin{center}
\begin{tabular}{ |l|l| }

\hline
\multicolumn{1}{|c|}{\multirow{3}{*}{$n$}} & \multicolumn{1}{|c|}{$D_{cor} $}\\
\cline{2-2}
&\multicolumn{1}{|c|}{100 intervals, logarithmically spaced range of $\varepsilon$ used}\\
\cline{2-2}
 &\rule{0pt}{12pt}  \hspace{85pt} $[\boldsymbol{e}^{-3.5},\boldsymbol{e}^{0}]$  \\
\hline
$3\times 10^{4}$ & \hspace{80pt} 1.884 $\pm$ 0.005  \\
\hline
$5\times 10^{4}$ & \hspace{80pt} 1.873 $\pm$ 0.005 \\
\hline
$6\times 10^{4}$ & \hspace{80pt} 1.888 $\pm$ 0.005 \\
\hline
\end{tabular}
\captionof{table}{Calculation of correlation dimension of Chua Double-Scroll attractor, least-square fittings\label{chuacor:tb}}
\end{center}
Calculation of the correlation integral in the linear region from \textbf{e$^{-3.5}$} to \textbf{e$^{0}$}, result in a correlation dimension $D_{cor}$ similar in magnitude to the other measures with as much number of points.
\vspace{20pt}
\subsubsection*{Conclusion}
While algorithms can be easily implemented in MATLAB\textregistered ~to calculate the measures, $N(\varepsilon)$, $S(\varepsilon)$ and $C(\varepsilon)$, inferring the different measures of dimension from the appropriate plots is not trivial, the linear region of the plots is not a well defined range and also local slopes are often present. The error estimates provided by that in the Least Square Method generally do not correspond to the actual error in the value obtained as drastically different values of slopes can be observed as different sizes of samples are used. However when a large sample can be used such methods allows to get a qualitative feel of the fractal dimension of the object.
\lstinputlisting[caption= Function file{,} L3capacitydimension.m]{Function_files/L3capacitydimension.m}
\lstinputlisting[caption= Function file{,} L3informationdimension.m]{Function_files/L3informationdimension.m}
\lstinputlisting[caption= Function file{,} L3correlationdimension.m]{Function_files/L3correlationdimension.m}
\medskip
\addcontentsline{toc}{section}{References}

%\begin{center}
%\begin{tabular}{ r r r r r}
%N & 1+0 & 1+1 & 1+2 & 1+4 \\
%50 & 0.002 & 0.107 & 0.130 & 0.169 \\
%500 & 0.007 & 0.038 & 0.038 & 0.044 \\
%5000 & 0.439 & 0.488 & 0.276 & 0.197\\
%50000 & 41.773 & 41.817 & 22.204 & 11.728\\		
%500000 & 3977.716 &	3990.536 & 2120.534 & 1157.357 \\
%\end{tabular}
%\end{center}
%%\includepdf{img/test2.pdf}
%
